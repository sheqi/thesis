\chapter{Bayesian Computation in Neural Circuits}

The final chapter will take a different view on Bayesian computation.
Rather than using probabilistic models and Bayesian inference to 
discover structure in neural recordings, we will consider the 
hypothesis that the neural circuits under study are actually 
implementing Bayesian computations. This ``Bayesian brain'' hypothesis 
is by no means novel --- it has been the subject of much recent 
research and debate in the computational neuroscience community. 
Under this hypothesis, the brain uses a probabilistic model 
of the world, and implements Bayesian inference algorithms to 
infer latent or missing variables under noisy observations. 
The probabilistic model is learned through experience, supervised 
training, or evolution, though the mechanisms of learning are 
less thoroughly explored. 

\section{Introduction} 
Consider a simple inference task that may be performed by hippocampal circuits. 
As a rat navigates an environment, it must constantly update its internal 
representation of its location. This is inherently a problem of inference: 
the rat has multiple sources of information to inform its judgment, observations
of salient environmental features, efferent copies of its motor commands, as
well as its past estimates of its location. Bayes' rule dictates how this 
information should be combined in order to compute a posterior distribution 
over location. By reasoning over the posterior distribution of locations 
rather than a single point estimate, the rat can take its uncertainty into 
account and make robust decisions. 

The ingredients of Bayesian reasoning are threefold. We start with a 
probabilistic model of the world that specifies the relationships between
random variables. For example, a conditional distribution over what cues 
are visible from a particular location. Next, we need an algorithm for 
drawing inferences about the posterior distribution of latent variables 
given observations. Finally, we need an algorithm for learning the 
model parameters from data. 

We begin by hypothesizing a probabilistic model that hippocampus could implement.
Our first question is how a posterior inference could be computed by hippocampal 
circuitry that has a limited number of neurons and a sparse set of connections.
These architectural constraints limit the fidelity with which a circuit can 
implement a Bayesian inference algorithm. We explore the trade-offs between 
neural architecture and inferential accuracy. Next, we consider how the 
parameters of this model could be learned from experience.

\section{Hippocampal Model}
We propose a simple probabilistic interpretation of hippocampal place cells.
It is well known that the hippocampus is involved in locomotory navigation.
Place cells in the hippocampus are selectively active when the rat is 
in a particular location in the environment. We suggest that this activity 
encodes a probability distribution over locations, and that the hippocampal 
circuitry is responsible for computing this probability distribution by 
combining environmental cues with its past estimates. 

To formalize this hypothesis, we propose that an environment is represented 
by a set of discrete locations,~${\mcL = \{1, \ldots, L\}}$. These locations 
need not be the same size in the real-world, so one location may correspond to
``the corner of the room'' whereas another may correspond to ``this precise 
spot.'' Indeed, how the rat chooses to divide the real world into discrete 
locations is an interesting question.

At any point in time, 
the rat's location is one of these discrete values,~${z_t \in \mcL}$. 
This location is not known to the rat, it must be inferred. 
The key information available to the rat is a set of environmental observations,
which we also take to be discrete~${\mcO = \{1, \ldots, O\}}$. At 
each instant in time, the rat receives an observation,~${x_t \in \mcO}$. 
We assume that the rat has learned a probabilistic model that specifies 
two conditional distributions: the distribution over current locations given 
previous locations,~${p(z_{t} \given z_{1:t-1})}$, and a distribution over observations 
for each location,~${p(x_t \given z_{t})}$. Moreover, we assume that this 
model is \emph{Markovian}, in that the current location only depends 
on the most recent location,
\begin{align}
p(z_t \given z_{1:t-1}) = p(z_t \given z_{t-1}) = A_{z_t, z_{t-1}},
\end{align}
where~$A \in [0,1]^{L \times L}$ is a \emph{transition} matrix whose rows sum to one.


Inference corresponds to computing the posterior distribution over 
locations given 
observations. Since there are only~$L$ different locations, the posterior distribution is a 
normalized vector,
\begin{align}
\bpi_t = \begin{bmatrix} p(z_t=1 \given x_{1:t}), & \ldots, & p(z_t=L \given x_{1:t}) \end{bmatrix}^\trans.
\end{align}
Similarly, denote the vector of observation likelihoods for each possible location by,
\begin{align}
{\bo_t = \begin{bmatrix} p(x_t \given z_t=1) & \ldots & p(x_t \given z_t=L) \end{bmatrix}^\trans}.
\end{align}
Then, 
\begin{align}
  \bpi_t 
  &= \sum_{\ell=1}^L p(z_{t}, z_{t-1} = \ell \given x_{1:t}) \\
  &= \frac{1}{\mcZ_t} p(x_t \given z_t) \sum_{\ell=1}^L p(z_{t} \given z_{t-1} = \ell) \, p(z_{t-1}=\ell \given x_{1:t-1}) \\
  &= \frac{1}{\mcZ_t} \bo_t \sum_{\ell=1}^L A_{z_t, \ell} \, \pi_{t-1, \ell} \\
  \label{eq:inf_calc}
  &= \frac{1}{\mcZ_t} \bo_t \odot \bA \bpi_{t-1},
\end{align} 
where we have used the factorization of the model to simplify the form
of the summation, and~$\odot$~denotes elementwise
multiplication. The scaling factor~$\mcZ_t$ ensures the distribution is 
properly normalized,
\begin{align}
\mcZ_t &= \sum_{\ell} \pi_{t,\ell} = \sum_{\ell} \sum_{\ell'} o_{t,\ell} \, A_{\ell, \ell'} \, \pi_{t-1, \ell'}.
\end{align}
Thus, the ideal computation only requires linear
operations and normalization. 
Our question is how well this computation can be
approximated within the limitations of neural circuits.


\section{Neural implementation} How could these calculations be 
carried out by hippocampal circuitry? First, we must hypothesize 
how~$\bpi_t$,~$\bo_t$, and~$\bA$ are instantiated in a circuit. 
Since place cells fire when the rat is in a particular location, 
we suggest that the spiking activity of place cells represents~$\bpi_t$.
This naturally suggests that~$\bA$, which conveys the probability of 
one location given past locations, should be implemented in the 
synaptic weights between place cells. Finally,~$\bo_t$, which 
reflects the likelihood of a location under the environmental observations, 
is likely represented by synaptic inputs to hippocampus from cortex.

To formalize this, suppose there are~$N$ place cells in
hippocampus. Each neuron is associated with a particular
location,~$\ell_n$. Moreover, suppose that there are~$R$ neurons that
represent each discrete location.  Since we make no assumptions about
the actual size of a location (in real world coordinates), the density
of place cells will differ for each discrete location.
We model the neurons as Poisson units, and we represent the instantaneous 
state of the population with an array,~${\bS_t \in \naturals^{L \times R}}$,
where the  spike,
\begin{align}
s_{t,\ell,r} &\sim \distPoisson(\lambda_{t, \ell, r}),
\end{align} denotes the number of action potentials
emitted by the~$r$-th neuron
representing location~$\ell$ fired at time~$t$.

While there have been many proposals of how neural activity may encode
probability distributions, here we assume a simple, direct encoding of
probability. Let,
\begin{align}
  \label{eq:pr_rep}
  \widehat{\pi}_{t, \ell} &= \frac{M_{t, \ell}}{M_t} &
  M_{t,\ell} &= \sum_{r=1}^R s_{t,\ell,r}, &
  M_t &= \sum_{\ell=1}^L M_{t,\ell}.
\end{align}

How accurately can probabilities be represented and recovered using 
such an encoding scheme? The accuracy is affected by two sources of error: 
the finite number of spikes, and the stochasticity of the spike 
counts. As for the former, the fidelity of this representation 
is a function of the total number of spikes,~$M_t$. As the number 
of spikes goes to zero, the number of different distributions that 
can be represented goes to zero as well. Conversely, as the spike 
count goes to infinity, we have infinite precision to represent 
distributions. In effect, representing a probability distribution 
with a finite number of spikes is equivalent to representing a 
real number with a finite precision computer. 

The stochasticity of the spike counts means that any instant in 
time, the probability distribution that is represented by the population will 
be a random variable.  First, we will show that this representation 
is unbiased. That is, if the firing rates,~$\lambda_{t, \ell, r}$, 
are proportional to the probability,~$\pi_{t,\ell}$, then the 
represented probability,~$\widehat{\pi}_{t, \ell}$, will have the  
correct expectation.

\begin{lemma}
\label{lem:consistency}
If the firing rates are proportional to a given
probability distribution then the probability distribution represented by the 
population will have expectation equal to the given probability distribution.
That is, if~$\lambda_{t, \ell, r}=\gamma_t \pi_{t,\ell}$, 
then~$\bbE[\widehat{\pi}_{t,\ell}] = \pi_{t,\ell}$.
\end{lemma}

\begin{proof}
  Iterating expectations, we have,
  \begin{align}
    \bbE[\widehat{\pi}_{t,\ell}] &=
    \bbE \left[ \frac{M_{t, \ell}}{M_{t}} \right] 
    = \bbE_{M_t} \left[
      \bbE_{M_{t,\ell}} \left[
        \frac{M_{t,\ell}}{M_t} \, \bigg|\,
        M_{t}  \right] \right].
  \end{align}
  Since the~$s_{t, \ell, r}$ are independent Poisson random variables, their partial
  sums are as well. Specifically,
  \begin{align}
    M_t &\sim \distPoisson \left( \sum_{\ell} \sum_{r} \lambda_{t,\ell,r} \right) \\
    &= \distPoisson \left(\gamma_t \sum_\ell \sum_r \pi_{t,\ell} \right) \\
    &= \distPoisson(\gamma_t \, R),
  \end{align}
  and
  \begin{align}
        M_{t,\ell} = \sum_r s_{t, \ell, r} &\sim \distPoisson \left( \gamma_t \, R \, \pi_{t, \ell} \right).
  \end{align}
  Moreover, their conditional distribution is binomial.
  \begin{align}
    M_{t,\ell} \given M_t &\sim
    \distBinomial \left( M_t, \frac{\gamma_t \, R \, \pi_{t,\ell}}{\gamma_t \, R} \right)
    =\distBinomial(M_t, \pi_{t,\ell}),
  \end{align}
  which has expectation~$\pi_{t,\ell} \, M_t$.
  Plugging this into the iterated expectation above, 
  \begin{align}
    \bbE[\widehat{\pi}_{t,\ell}]
    &= \bbE_{M_t} \left[
      \bbE_{M_{t,\ell}} \left[
        \frac{M_{t,\ell}}{M_t} \, \bigg|\,
        M_t  \right] \right] \\
    &= \bbE_{M_t} \left[ 
      \frac{\pi_{t,\ell} M_t}{M_t} \right]\\
    &= \pi_{t,\ell}.
  \end{align}
  Thus, this procedure is unbiased.
\end{proof}

While this stochastic representation may be correct in expectation, we
would like to characterize the probability that it is ``close'' to its
mean. As we hypothesized above, the difference between the true
probability and that represented by the population should shrink as
the number of spikes grows. We quantify this with the following
theorem, which provides an upper bound on the number of spikes
required to guarantee that the represented probability differs from
the true probability by more than~$\epsilon$. We measure this
difference with the total variation distance between the
distributions,
\begin{align}
  \dtv(\widehat{\bpi}_t, \bpi_t) &= 
  \max_{\ell} |\widehat{\pi}_{t,\ell} - \pi_{t,\ell}|.
\end{align}

\begin{theorem}
Given~$\lambda_{t,\ell,r} = \gamma_t \pi_{t,\ell}$, a fixed 
error level~$\epsilon < 1$, and a desired confidence~$\delta < 1$, 
there exists a minimum number of spikes~$M^*$ such that if~$M_t \geq M^*$,
the conditional probability of error is bounded 
by~$\Pr(\dtv(\widehat{\bpi}_t, \bpi_t) > \epsilon \given M_t) < \delta$.
Furthermore, this minimum number of spikes is at most,
\begin{align}
M^* &\leq \frac{3}{\epsilon^2} \ln \frac{2L}{\delta},  
\end{align}
\end{theorem}

\begin{proof}
First, consider the probability that a particular entry differs from its mean by more than~$\epsilon$.
\begin{align}
  &\Pr(|\widehat{\pi}_\ell - \pi_\ell |  > \epsilon \given M_t) \\
  &\qquad= \Pr(\widehat{\pi}_\ell - \pi_\ell  > \epsilon \given M_t) + 
  \Pr(\widehat{\pi}_\ell - \pi_\ell  < -\epsilon \given M_t) \\
  &\qquad= \Pr \left(M_{t, \ell} > M_t \pi_{t, \ell} \left(1+\frac{\epsilon}{\pi_{t, \ell}} \right) \, \bigg| \, M_t \right) 
  +\Pr \left(M_{t, \ell} < M_t \pi_{t, \ell} \left(1-\frac{\epsilon}{\pi_{t, \ell}} \right) \, \bigg| \, M_t\right) \\
  &\qquad= \Pr \left(M_{t, \ell} > \bbE[M_{t,\ell} \given M_t] \left(1+\frac{\epsilon}{\pi_{t, \ell}} \right) \right) 
   +\Pr \left(M_{t, \ell} < \bbE[M_{t,\ell} \given M_t] \left(1-\frac{\epsilon}{\pi_{t, \ell}} \right) \right)
\end{align}
As in Lemma~\ref{lem:consistency}, we have used the fact
that~$M_{t, \ell} \given M_t \sim \distBinomial(M_t, \pi_{t, \ell})$,
and hence has expectation~$M_t \pi_{t, \ell}$. 
The probability of this binomial random variable exceeding its mean 
by a multiplicative constant is a decreasing function of the 
number of trials,~$M_t$. This implies that there exists a minimum 
number of trials~$M^*$ such that for~$M_t \geq M^*$, this probability 
of error is bounded above by~$\delta$, hence proving the first part 
of the theorem.

Now suppose~$M_t=M^*$. Since a binomial random variable,~$M_{t,\ell}$, can be seen
as a sum of independent coin flips, we can use a Chernoff bound to
upper bound the probability of deviating from the mean by a
multiplicative factor. We
have,
\begin{align}
  \Pr \left(M_{t, \ell} > \bbE[M_{t,\ell} \given M^*] \left(1+\frac{\epsilon}{\pi_{t, \ell}} \right) \right) 
  &\leq \exp \left \{- \frac{M^* \epsilon^2}{3 \pi_{t, \ell}} \right \},
\end{align}
and
\begin{align}
  \Pr \left(M_{t, \ell} < \bbE[M_{t,\ell} \given M^*] \left(1-\frac{\epsilon}{\pi_{t, \ell}} \right) \right) 
  &\leq \exp \left \{- \frac{M^* \epsilon^2}{2 \pi_{t, \ell}} \right \}.
\end{align}
Combining these, and leveraging the fact that~$\pi_{t, \ell} \leq 1$, we get,
\begin{align}
  \Pr(|\widehat{\pi}_{t,\ell} - \pi_{t,\ell} |  > \epsilon \given M^*) 
  & \leq 2 \exp \left \{- \frac{M_t \epsilon^2}{3} \right \}.
\end{align}
We bound the maximum deviation of any entry in~$\widehat{\bpi}$ with a union bound,
\begin{align}
  \Pr(\dtv(\widehat{\bpi}_t, \bpi_t) > \epsilon \given M^*) 
  & \leq 2L \exp \left\{-\frac{M^* \epsilon^2}{3} \right\}.
\end{align}
Setting this probability equal to~$\delta$ yields the desired bound on~$M^*$,
\begin{align}
  M^* &\leq \frac{3}{\epsilon^2} \ln \frac{2L}{\delta}.
\end{align}

\end{proof}


This theorem provides an upper bound on the minimum number of spikes necessary to 
guarantee that the total variation distance between the true and estimated 
probability vectors is less than~$\epsilon$ with probability~$1-\delta$. 
Notably, the relevant quantity is the number of spikes~$M_t$, rather than 
the number of neurons. Thus, there is some flexibility in how the probability 
is estimated: a small population of neurons could be measured over many time bins,
or a large population could be measured over a single time bin. Moreover, the 
population gain,~$\gamma_t$, could be varied to adjust the number of spikes 
per time bin. 



 
\begin{comment}
\subsection{Neural dynamics that perform inference}

We show how a hippocampal circuit could perform approximate inference using 
simple neural update rules. 

Plugging Eq.~\ref{eq:pr_rep} into Eq.~\ref{eq:inf_calc}, our update rule should set,
\begin{align}
  \label{eq:pop_update}
  M_{t,\ell} 
  &\propto o_{t,\ell} \sum_{\ell'=1}^L A_{\ell, \ell'} \, M_{t-1, \ell'}
  = o_{t, \ell} \sum_{\ell'=1}^L \sum_{r'=1}^R A_{\ell, \ell'} \, s_{t-1, \ell', r'}
  %&= \frac{1}{\mcZ_t} o_{t,\ell} \sum_{\ell'=1}^L  \sum_{r'=1}^R \frac{A_{\ell, \ell'}}{R} \, s_{t-1,\ell',r'},
\end{align}


It is straightforward to design a stochastic update rule that 
implements this inference computation in expectation. 
For each neuron~$(\ell,r)$, in parallel, sample its state according to,
\begin{align}
  s_{t,\ell,r} &\sim \distPoisson (\lambda_{t, \ell, r}) \\
  \label{eq:rates}
    \lambda_{t, \ell, r} &= 
    \gamma_t \, o_{t,\ell} \sum_{\ell'=1}^L \sum_{r'=1}^R A_{\ell, \ell'} \cdot s_{t-1,\ell',r'},
\end{align}
where~$\gamma_t$ is a tunable gain parameter. In expectation,
\begin{align}
  \bbE[M_{t, \ell}] = \sum_{r=1}^R \lambda_{t, \ell, r}
  = R \, \gamma_t \, o_{t,\ell} \sum_{\ell'=1}^L \sum_{r'=1}^R A_{\ell, \ell'} \cdot s_{t-1,\ell',r'}
  \propto o_{t,\ell} \sum_{\ell'=1}^L A_{\ell, \ell'} \, M_{t-1, \ell'},
\end{align}
as required by Eq.~\ref{eq:pop_update}. As we will show below, the
expectation of~$\widehat{\pi}_{t,\ell}$, it is also true that,
\begin{align}
  \bbE[\widehat{\pi}_{t,\ell}] &= \frac{o_{t,\ell} \sum_{\ell'=1}^L A_{\ell, \ell'} \, M_{t-1, \ell'}}{\sum_{j} o_{t,j} \sum_{\ell'=1}^L A_{j, \ell'} \, M_{t-1, \ell'}}.
\end{align}

\subsubsection{Sparse Connectivity}
The rate calculations in Eq.~\ref{eq:rates} assume that the neural circuit
is completely connected, whereas we know these circuits are actually quite sparse.
To model this, we introduce a graph,~${G \in \{0,1\}^{LR \times LR}}$.
If~$G_{(\ell,r),(\ell',r')}=1$, then there exists a directed connection from
neuron~$(\ell',r')$ to neuron~$(\ell, r)$. Only connected neurons are
included in the rate equation:
\begin{align}
    \lambda_{t, \ell, r} &= 
    \gamma_t \, o_{t,\ell} \sum_{\ell'=1}^L \sum_{r'=1}^R
    G_{(\ell,r),(\ell',r')} \cdot A_{\ell, \ell'} \cdot s_{t-1,\ell',r'},
\end{align}

We start by considering a simple graph model in which each entry in~$G$
is an independent Bernoulli random variable with probability~$\rho$.
If a connection exists,
we assume it has the correct weight,~$A_{\ell, \ell'}$.
This induces additional randomness
in the rate function. The expectation is
still proportional to the desired update.
\begin{align}
  \bbE_G[\lambda_{t, \ell, r} \given \bS_{t-1}]
  &= \rho \, \gamma_t \, o_{t, \ell}
  \sum_{\ell'} A_{\ell, \ell'} \cdot s_{t-1, \ell', r'}.
\end{align}
There is, however, additional variance,
\begin{align}
  \Var_G(\lambda_{t, \ell, r} \given \bS_{t-1})
  &= \rho(1-\rho) \, \gamma_t^2 o_{t,\ell}^2
  \sum_{\ell'} A_{\ell, \ell'}^2 \cdot s_{t-1, \ell', r'}^2 .
\end{align}
We will analyze the effects of this variance on the variance
of the resulting probability estimates. 

\subsubsection{Normalization}
To prevent the number of action potentials from shrinking to zero 
or saturating with all neurons active, we need a mechanism for normalizing the 
population activity. 

(todo: update/formalize this for Poisson) From an information theoretic perspective, it is best to
have~${M_t = R}$ neurons active at any time. Since there are~$R$ neurons per location, 
this means we can represent any probability,~${\widehat{\pi}_{t,\ell}=r/R}$ for~$r \in \{0,\ldots,R\}$.
With fewer active neurons, we have a coarser discretization; with more, we can 
only represent probabilities less than one.

If our goal is to set~$M_t=R$, then we should strive for,
\begin{align}
  R \approx \mathbb{E}[M_t] &= \gamma_t \sum_{\ell} \sum_{r} o_{t,\ell} \sum_{\ell'=1}^L \sum_{r'=1}^R A_{\ell, \ell'} \cdot s_{t-1,\ell',r'} \\
  &= R \gamma_t \sum_{\ell} o_{t,\ell} \sum_{\ell'=1}^L \sum_{r'=1}^R A_{\ell, \ell'} \cdot s_{t-1,\ell',r'} \\
\implies \gamma_t &= \left( \sum_{\ell} o_{t,\ell} \sum_{\ell'=1}^L \sum_{r'=1}^R A_{\ell, \ell'} \cdot s_{t-1,\ell',r'} \right)^{-1}
\end{align}
While this cannot be computed in a causal manner, we can impose slower
temporal dynamics on~$s$ such that~$M_{t} \approx M_{t-1}$,
and then set~$\gamma_t = R / M_{t-1}$. (todo: more on this later)

\section{Analyzing the Updates}
There are at least three sources of stochasticity that can affect the
accuracy of the inferences. The first is that the probability
distribution is represented by a finite population of stochastic
neurons. Even if those neurons are driven with firing rate
proportional to their exact probability, the sampled probability will
still be a random variable with some variance. Second, the neurons are
only sparsely connected, so their rates will have additional variance
due to the subsampled estimates of~$\widehat{\pi}_{t-1,\ell'}$ used in
the updates. The final source of stochasticity arises because the
rates are a function of the spike counts in the preceding time bin.
Thus, the spike counts follow a stochastic process that may or may not
be stable.


\subsection{Stability of the stochastic process}
The dynamics of Eq.~\ref{eq:rates} define a discrete time Hawkes process.
Written in matrix form, the dynamics are,
\begin{align}
  \blambda_t &= (\gamma_t \bo_t \bone^\trans \odot \bG \odot \bA) \bs_{t-1}
  \; & \;
  \bs_t &\sim \distPoisson(\blambda_t).
\end{align}
The stability of this system is intimately tied to the eigenvalues
of~${\gamma_t \bo_t \bone^\trans \odot \bG \odot \bA}$.
(todo: double check for discrete time) In the case where~$\gamma_t$
and~$\bo_t$ are static, the system will be stable if all eigenvalues lie
inside the complex unit disk. In fact, unless there is an eigenvalue
exactly equal to one, the stable system will eventually decay to having
zero spikes. This actually motivates a baseline firing rate~$b$, such that,
\begin{align}
    \blambda_t &= b + (\gamma_t \bo_t \bone^\trans \odot \bG \odot \bA) \bs_{t-1}
\end{align}
(todo: does this bias the inferences? probably induces a bias-variance tradeoff!)

Assuming stationarity and static~$\bo_t$ and~$\gamma_t$,
\begin{align}
  \blambda_{\textsf{ss}} &= \bbE[\blambda_t] 
  = b\bone + (\gamma_t \bo_t \bone^\trans \odot \bG \odot \bA) \blambda_{\textsf{ss}}
\end{align}
which implies,
\begin{align}
  \blambda_{\textsf{ss}} &=
  b \left( \bI - \gamma \bo \bone^\trans \odot \bG \odot \bA \right)^{-1} \bone
\end{align}

To start, assume~$\gamma=1$,~$\bo=\bone$,
and~${\bG = \bone\bone^\trans}$ (fully connected) so that
${\blambda_{\textsf{ss}} = b \left( \bI - \bA \right)^{-1} \bone}$.
Under these conditions, the model is receiving uninformative
observations, so we expect the posterior probability to be determined
solely by the location dynamics model. How do we reconcile this 
steady state solution with our intuition from Markov chain 
theory that the steady state should converge to the principal 
eigenvector of the transition matrix? If~$\bA$ is a row 
stochastic transition matrix (all rows sum to one), 
then by the Perron-Frobenius theorem the maximum eigenvalue is 
exactly one. Thus,~$\bI - \bA$ will be singular, and its inverse 
will diverge to infinity. However, in Markov chains,~$b=0$, 
so this divergence is canceled out and the limit is 
the principal eigenvector of~$\bA$.

Can we design a system with~$b>0$ and~$\rho(\bA)<1$ that closely 
approximates this steady state distribution? 
\end{comment}