% \begin{savequote}[75mm]
% Nulla facilisi. In vel sem. Morbi id urna in diam dignissim feugiat. Proin molestie tortor eu velit. Aliquam erat volutpat. Nullam ultrices, diam tempus vulputate egestas, eros pede varius leo.
% \qauthor{Quoteauthor Lastname}
% \end{savequote}

\chapter{Nonlinear Interaction Networks}



Hawkes process inference relied on an augmentation strategy 
made possible by the linear form and excitatory nature 
 of the interactions. In neural settings, these assumptions 
of the Hawkes process are not as realistic. Instead, we turn to 
the nonlinear Hawkes process, which allows for both excitatory 
and inhibitory interactions by introducing a nonlinearity 
into the model. This corresponds to the popular generalized 
linear model that is widely used in computational neuroscience. 
We develop a novel approach that leverages the 
\polyagamma augmentation to enable efficient, fully-conjugate 
inference. Again, we combine the nonlinear Hawkes process with 
prior distributions on the network of interactions, and we focus 
on a discrete time approximation. This is part of ongoing work 
with Ryan and Jonathan Pillow at Princeton, and it has grown out 
of a Cosyne abstract \cite{linderman2015cosyne}. I also have an 
unpublished draft in which I extend the abstract and apply it 
to a variety of neural datasets, and show how the network models 
recover interesting latent features like neural types and locations.
I will submit a journal version of this paper as soon as possible.

\section{Model}

\section{MAP Inference}

\section{P\'{o}lya-gamma Augmentation}

\section{Bayesian Inference with MCMC}

\section{Variational Inference}
