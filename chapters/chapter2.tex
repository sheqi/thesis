% \begin{savequote}[75mm]
% This is some random quote to start off the chapter.
% \qauthor{Firstname lastname}
% \end{savequote}

\chapter{Hawkes Processes with Latent Network Structure}

This chapter will draw on two papers, my 2014 ICML paper \cite{linderman2014discovering}, 
which shows how to do Gibbs sampling in a continuous time model,
and my arXiv preprint \cite{linderman2015scalable}, which shows 
how to do Gibbs sampling and stochastic variational inference in 
a discrete time approximation.  Both papers focus on
adding prior distributions on the network of interactions, and 
showing that the latent variables of the network models can be 
meaningful, and that they can aid in prediction of future 
activity. 

Hawkes processes are a particularly nice starting point 
because they capture a salient type of structure --- excitatory 
interactions between nodes --- while retaining some 
analytical tractability due to the linear form of the 
interactions. In addition to deriving inference algorithms, I 
will also show how we can tailor our prior distributions to 
maintain stability of the network. 

\section{Model}

\subsection{Augmentation with Parent Variables}

\subsection{Background Models}

\subsection{Process Identity Models}

\section{MAP Inference}

\section{Bayesian Inference with MCMC}

\section{Variational Inference}

\subsection{Mean Field Approximation}

\subsection{Stochastic Variational Inference}

\section{Experiments}



