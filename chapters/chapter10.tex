\chapter{Conclusion}
\label{conclusion}

% Preamble
% - success depends on merging top-down and bottom-up approaches
Recent advances in neural recording technologies have stirred great
excitement. It seems that a more complete understanding of neural
computation is within our grasp. Armed with these powerful tools, we
can peer into the brain and observe the activity of most, if not all,
of the neurons in a circuit. What a major advance over the handfuls
of neurons we were limited to only a few years ago! All we must do
is extract the underlying patterns and principles from these
large scale recordings.

While these advances present unprecedented opportunities, the task of
translating data into understanding is far from trivial. The human
genome has been known for a decade now \citep{international2004finishing, gregory2006dna}, yet much of its structure
remains enigmatic. The connectome of the nematode \textit{C. Elegans}
has been known for three decades \citep{white1986structure}, yet our
understanding of this simple organism with fewer than 400 neurons is
still incomplete. At the heart of these endeavors is the search for
meaningful abstractions and structured representations given complex
and noisy data. Our success in reverse engineering neural computation
relies critically on our ability to discover such structure. This
thesis has developed a number of methods for instantiating structural
hypotheses in the form of probabilistic models and turning the crank
of Bayesian inference in order to reason about them.

% Why should we expect to find such structure in the first place?  A
% priori, it is not clear that the brain should have a simple design. We
% have certainly not evolved under selective pressure for ease of
% decipherability, but there are still many reasons to remain
% hopeful. For one, we have already found many beautiful and
% surprisingly simple patterns. Hippocampal place cells and the response
% properties of the early visual system are classic examples, as is the
% coding of reward prediction error by dopamine neurons in the ventral
% tegmental area and substania
% nigra~\citep{schultz1997neural}. Similarly simple explanations have
% been found at the cognitive level, like the Bayesian theories of the
% last chapter.

% A second cause for hope is that the brain exhibits highly stereotyped and
% modular patterns of connectivity \citep{shepherd2003synaptic}.  That
% the underlying ``hardware'' is conserved across brain regions
% suggests that the algorithms that run on this hardware may also be shared
% by a diverse assortment of computations. Rather than searching for
% unique circuits and algorithms for each individual task, we hope for
% a few general purpose algorithms that are broadly deployed.

% Finally, while the brain may not have evolved for ease of reverse
% engineering, we have evolved a general-purpose architecture that
% supports learning and reasoning.  While much of our mental faculty is
% inherited through our genetic code, a great deal is learned from
% experience. Moreover, our present day experiences differ tremendously
% from those of our not-so-distant ancestors, yet we continue to adapt
% and thrive in the face of evolutionarily unforseen
% circumstances. These all point to general purpose algorithms
% underlying human intelligence rather than bespoke algorithms for
% every problem.

I believe the path forward lies in the iterative refinement of
theories guided by both top-down considerations of algorithmic goals
and complexity-theoretic constraints, and bottom-up, data-driven
analyses of neural data. The Bayesian methods presented in this thesis
are designed to accelerate this process, providing ``data
microscopes'' that allow us to visualize complex, high-dimensional
data in new and interpretable ways. I will briefly discuss two
directions in which these methods should continue to be developed.

\section{Toward Programmatic Models of Neural Computation}
% Programmatic models
% - program induction
% - reinforcement learning
% - general purpose inference algorithms to go with
%What type of abstraction or description should we seek?  There is no
%right answer to this question: our understanding of neural computation
%must admit descriptions at multiple levels of adstraction.
As we have
shown here, hierchical probabilistic models provide an intuitive
language for capturing different types of abstraction, allowing us to
formalize generative processes of how data comes to be. However, as
these models grow in scope and scale, the language of probabilistic
models becomes cumbersome. At the same time, as our models grow in
complexity, they look more and more like \emph{probabilistic
  programs} \citep{goodman2008church}.

The probabilistic models developed in this thesis can all be written in
this way. For example, the hidden Markov models of
Chapter~\ref{chap:seven} are equivalent to the following program:

\begin{center}
  \begin{minipage}{.75\textwidth}
    \vskip-2em
    \singlespacing
    \begin{algorithm}[H]
      \floatname{algorithm}{Program}
      \sffamily
      \begin{algorithmic}
        \Require $\bpi^{(0)}$,~$\bP$,~$\bLambda$
        \For {$t = 1, \ldots, T$}
          \If {$t=1$}
            \State $z_t \sim \distDiscrete(\bpi^{(0)})$
          \Else
            \State $z_t \sim \distDiscrete(\bpi^{(z_{t-1})})$
          \EndIf
          \For {$n = 1, \ldots, N$}
            \State $s_{t,n} \sim \distPoisson(\lambda_{z_{t},n})$
          \EndFor
        \EndFor
      \end{algorithmic}
      \caption{Programmatic representation of a hidden Markov model.}
      \label{prog:hmm}
    \end{algorithm}
  \end{minipage}
\end{center}

This representation is equivalent to the probabilistic model (it
implies the same distribution over~$\bz$ and~$\bS$), but this
description combines stochastic operations, like sampling,
with basic control flow, like \textsf{\textbf{if}}
statements and \textsf{\textbf{for}} loops.  This powerful combination
not only enables rapid formulation of models for neural data, it also
forms the basis for a ``probabilistic language of thought,'' an idea
that is taking hold in cognitive science \citep{goodman2014concepts}.
As we seek to bridge the gap between cognitive algorithms and neural
implementations, it will help if we are speaking the same language.

Of course, the ``no free lunch'' theorem applies here as well. While
probabilistic programming languages make it easy to specify complex
generative processes, they make it just as easy to specify models for
which Bayesian inference is completely intractable. While much
progress has been made in general purpose inference algorithms
\citep{goodman2008church, ranganath2013black, mansinghka2014venture,
  wood2015new, kucukelbir2015automatic}, these ``black box'' inference
algorithms are, by design, not capitalizing on model-specific
structure that the rather bespoke inference algorithms of this thesis
have leveraged. This will certainly change as probabilistic program
``compilers'' become more adept at recognizing model structure, but
this is currently a major challenge.


\section{Toward Joint Models of Neural Activity, Behavior, and Environment}
% Joint models of behavior/stimuli and neural activity
% - state space models of behavior
This thesis has focused solely on modeling the dynamics and structure
of neural spike trains, however, this data is often collected from
organisms as they perform natural behaviors in complex behaviors. For
example, massive recordings are now being collected from animals in
decision making \citep[e.g.][]{briggman2005optical}, freely behaving
\citep[e.g.][]{prevedel2014simultaneous}, and evoked response
\citep[e.g.][]{portugues2014whole} tasks.  This type of data provides
a tremendous opportunity to study the relationship between neural
activity, natural behavior, and environment. But first, we must
formulate and fit a model that captures both the complex dynamics of
neural activity, the rich repertoire of behavior, and the
environmental state. The models and inference algorithms designed in
this thesis capture core notions of state and dynamics that can be
extended, in an intuitive way, to these types of recordings. 

For example, large-scale recordings have revealed that ensembles of
neurons reliably participate together during natural or trained
behavior, suggesting that task-related neural activity might be
lower-dimensional than the number of recorded neurons, and that these
neurons might evolve through different states over time in an
environmentally dependent manner.  The dynamical system models
developed in this thesis can naturally instantiate these
hypotheses. Consider a model in which neural spike trains,~$\bs_t$,
behavioral time series,~$\bb_t$, and environmental stimuli,~$\be_t$,
are simultaneously measured. A simple hypothesis is that the neural
spike trains and the behavior are conditionally independent given
underlying states,~$\bx_t$ and~$z_t$, and that the evolution of these
states depends on the environment,~$\be_t$. Such a model enables us to
identify the low dimensional states of neural activity and overt
behavior, as well as their dynamics. Moreover, it enables us to
predict one given the others. With relatively minor adjustments, the
inference algorithms developed previously can be extended to handle
these multimodal datasets.

Ultimately, the goal of computational and systems neuroscience is to
understand this interplay between environment, neural activity, and
behavior. As with all scientific endeavors, our success will be
measured in our ability to articulate theories of neural computation
that explain, in simpler terms, the complex nature of these
multifaceted systems.  With the advent of recording technologies
capable of probing neural circuits at unprecedented scale and
advances in machine learning providing the computational and
statistical tools for making sense of complex data, it seems the stage
is set for major breakthroughs in our understanding of nature's most
sophisticated computer: the human brain. 


