\chapter{Bayesian Computation in Neural Circuits}

The final chapter will take a different view on Bayesian computation.
Rather than using probabilistic models and Bayesian inference to 
discover structure in neural recordings, we will consider the 
hypothesis that the neural circuits under study are actually 
implementing Bayesian computations. This ``Bayesian brain'' hypothesis 
is by no means novel --- it has been the subject of much recent 
research and debate in the computational neuroscience community. 
Under this hypothesis, the brain uses a probabilistic model 
of the world, and implements Bayesian inference algorithms to 
infer latent or missing variables under noisy observations. 
The probabilistic model is learned through experience, supervised 
training, or evolution, though the mechanisms of learning are 
less thoroughly explored. 

\clearpage
\section{Outline}
Why be probabilistic? Distinguish between explicit representation and
computation of probability distributions and methods which learn
mappings from inputs to decisions without representing uncertainty. 

A theory of probabilistic computation must address at least three questions?
\begin{enumerate}
\item How do we represent probability distributions?
\item How do we perform inference? That is, how do we update these
  probabilities based on evidence?
\item How do we learn the probabilistic dependencies between random variables?
\end{enumerate}


What are the constraints facing this theory? Can we evaluate the
theoretical complexity in terms of number of neurons, number of
connections, etc?

What does neural activity look like? How can we use the work of
previous chapters to discover neurons for each variable and value?

\section{Introduction}
Bayesian theories of neural computation address a fundamental
question: how do organisms reason, act, and make decisions given only
limited, noisy information about the world around them?  Bayes' rule
tells us how an optimal agent should combine noisy information with
prior knowledge to make inferences and decisions under uncertainty. A
burgeoning body of evidence shows that organisms perform
near-optimally in many behavioral tasks that require reasoning under
uncertainty \todo{cite}, which suggests that the brain may be performing, or at
least approximating, Bayesian inference.  Given this evidence, it is
natural to ask what algorithms and neural implementations underly this
capability. 

Any theory that claims to answer to this question must address four
interrelated concerns.  First, it must specify how posterior
probabilities are represented, and how the conditional probability
distributions that constitute the probabilistic model are encoded.
Second, it must describe a set of neural dynamics that compute the
desired posterior distribution for a given set of observations. That
is, it must define an algorithm for probabilistic inference. These
dynamics must respect the natural constraints of neural systems, for
example, that neural connectivity is sparse and that neurons have
limited computational power. Finally, a theory is incomplete without a
description of how the parameters that define the conditional
probability distributions are learned, and how new variables of
interest are incorporated into an existing model.

\TODO{Discuss ways of evaluating theories: complexity and consistency
  with neural data.}

The purpose of this work is not to present a radically novel
  way of representing distributions, performing inference, or even
  learning probabilistic relationships. Indeed, the theory we present
  borrows much from existing work on these topics. Instead, our main contributions
  are
  \begin{enumerate}
    \item an analysis of the complexity of the proposed representation
  and algorithms, which provides nontrivial constraints on the
  biological realizations of this theory; 
  and \item a study of the observable consequences of this theory and
  a method for testing this theory with neural recordings. 
  \end{enumerate}


\section{Representation of probability distributions}
\TODO{Quickly survey previous approaches. Leave the longer conversation for the discussion.}

\sloppy
Perhaps the simplest representation of probability is a \emph{direct}
representation in which neural firing rates reflect instantaneous
probabilities. Assume a population of neurons is responsible for
representating the distribution over values that a set of random
variables may take on. We denote this set of variables 
by,~${\bz = \{z_1, \ldots, z_J\}}$.  For
simplicity, assume for now that these variables can only assume a
discrete set of values,~${\{1, \ldots, K\}}$.  \todo{example} Our
neural population is thus tasked with representing probabilty
vectors,~$\bpi^{(j)}$, for each 
variable. The entries of these vectors are,~${\pi^{(j)}_k = \Pr(z_j=k)}$.
\TODO{Later, we will discuss how this representation could be extended
  to allow continuous random variables.}

In a direct representation, each variable-value pair,~$(z_j,k)$, is associated with a population of neurons. The firing
rate of these neurons reflect the instantaneous probability of the
variable taking on that particular value.  Suppose that each
pair, is represented by~$R$ neurons. Moreover, suppose
these representations are \emph{non-overlapping} such that each neuron
can be represented with at most one variable-value pair.
Let~${j_n \in \{1, \ldots, H\}}$
denote the index of the specific variable.
Then, let~$k_n \in \{1, \ldots K\}$ denote the particular value that neuron~$n$ represents.
%A representation size of~$R$ implies that~$\sum_{n=1}^N \bbI[j_n=h] \, \bbI[k_n=k] = R$.

We assume these neurons are stochastic, each endowed
with an instantaneous firing rate,~$\lambda_{t,n}$, which gives rise to an
instantaneous spike count,~$s_{t,n}$ according to a Poisson distribution,
\begin{align}
s_{t,n} &\sim \distPoisson(\lambda_{t,n}).
\end{align}
These spike counts encode
instantaneous probability distributions for each variable,
\begin{align}
  \widehat{\pi}_{t,k}^{(j)} &=
  \frac{\sum_{n=1}^N \bbI[j_n=j] \, \bbI[k_n=k] \, s_{t,n}}
       {\sum_{n=1}^N \bbI[j_n=j] \, s_{t,n}}.
\end{align}

Even without any knowledge of the dynamics that gave rise
to these instantaneous firing rates and to the
encoded probability distribution, we can evaluate the complexity
of this representation. How many neurons, time steps, or spikes
are required to guarantee that the encoded distribution is within
some tolerable error of the underlying ground truth? The answer
to this question can provide some important constraints on the
viability of this approach and aid in our search for neural
susbtrates of inference. 

\section{Complexity of Direct Representations}
The stochasticity of the spike counts means that any instant in 
time, the probability distribution that is represented by the population will 
be a random variable.  First, we will show that this representation 
is unbiased. That is, if the firing rates,~$\lambda_{t, n}$, 
are proportional to the probability,~$\pi_{t,k_n}^{(j_n)}$, then the 
represented probability,~$\widehat{\bpi}_{t}^{(j)}$, will have the  
correct expectation.

\begin{lemma}
\label{lem:consistency}
If the firing rates are proportional to a given
probability distribution then the probability distribution represented by the 
population will have expectation equal to the given probability distribution.
That is, if~$\lambda_{t, n}=\gamma \pi_{t,k_n}^{(j_n)}$, 
then~$\bbE[\widehat{\pi}_{t,k}^{(j)}] = \pi_{t,k}^{(j)}$.
\end{lemma}

\begin{proof}
  Since we are focusing on the representation of a single random variable
  $z_j$, we drop the superscript~$(j)$ for this section. Let,
  \begin{align}
    S_{t,k} &= \sum_{n=1}^N \bbI[j_n=j] \, \bbI[k_n=k] \, s_{t,n},
  \end{align}
  and
  \begin{align}
    S_{t} &= \sum_{n=1}^N \bbI[j_n=j] \,  s_{t,n} = \sum_{k=1}^K S_{t,k}.
  \end{align}
  Iterating expectations, we have,
  \begin{align}
    \bbE[\widehat{\pi}_{t,k}] &=
    \bbE \left[ \frac{S_{t, k}}{S_{t}} \right]
    = \bbE \left[
      \bbE \left[
        \frac{S_{t,k}}{S_t} \, \bigg|\, S_{t}  
      \right] \right].
  \end{align}
  Since~$s_{t, n}$ are independent Poisson random variables, their partial
  sums are as well. Specifically,
  \begin{align}
    S_t &\sim \distPoisson \left( \sum_{n} \bbI[j_n=j] \lambda_{t,n} \right) \\
    &= \distPoisson \left(\gamma \sum_n \bbI[j_n=j] \sum_{k} \bbI[k_n=k] \, \pi_{t,k} \right) \\
    \label{eq:S_t_rate}
    &= \distPoisson(\gamma R),
  \end{align}
  and, similarly,
  \begin{align}
        S_{t,k} &\sim \distPoisson \left( \gamma \, R \, \pi_{t, k} \right).
  \end{align}
  Moreover, their conditional distribution is binomial.
  \begin{align}
    S_{t,k} \given S_t &\sim
    \distBinomial \left( S_t, \frac{\gamma \, R \, \pi_{t,k}}{\gamma \, R} \right)
    =\distBinomial(S_t, \pi_{t,k}),
  \end{align}
  which has expectation~$\pi_{t,k} \, S_t$.
  Plugging this into the iterated expectation above, 
  \begin{align}
    \bbE[\widehat{\pi}_{t,k}]
    &= \bbE \left[
      \bbE \left[
        \frac{S_{t,k}}{S_t} \, \bigg|\, S_t  \right] \right] \\
    &= \bbE \left[ 
      \frac{\pi_{t,k} S_t}{S_t} \right]\\
    &= \pi_{t,k}.
  \end{align}
  Thus, this procedure is unbiased.
\end{proof}

While this stochastic representation may have the correct expectation, we
would like to characterize the probability that it is ``close'' to its
mean. As we hypothesized above, the difference between the true
probability and that represented by the population should shrink as
the number of spikes grows. We quantify this with the following
theorem, which provides an upper bound on the number of spikes
required to guarantee that the represented probability differs from
the true probability by more than~$\epsilon$. We measure this
difference with the total variation distance between the
distributions,
\begin{align}
  \dtv(\widehat{\bpi}_t, \bpi_t) &= 
  \max_{\ell} |\widehat{\pi}_{t,k} - \pi_{t,k}|.
\end{align}

\begin{theorem}
Given a fixed probability vector~$\bpi_t$,
firing rates~$\lambda_{t,n} = \gamma \pi_{t,k_n}^{(j_n)}$, a fixed 
error level~$\epsilon < 1$, and a desired confidence~$\delta < 1$, 
there exists a minimum number of spikes~$S^*$ such that if~$S_t \geq S^*$,
the conditional probability of error is bounded 
by~$\Pr(\dtv(\widehat{\bpi}_t, \bpi_t) > \epsilon \given S_t) < \delta$.
Furthermore, this minimum number of spikes is at most,
\begin{align}
S^* &\leq \frac{1}{2\epsilon^2} \ln \frac{2K}{\delta},  
\end{align}
\end{theorem}

\begin{proof}
First, consider the probability that a particular entry differs from its mean by more than~$\epsilon$.
\begin{align}
  &\Pr(|\widehat{\pi}_k - \pi_k |  > \epsilon \given S_t) \\
  &\qquad= \Pr(\widehat{\pi}_k - \pi_k  > \epsilon \given S_t) + 
  \Pr(\widehat{\pi}_k - \pi_k  < -\epsilon \given S_t) \\
  &\qquad= \Pr \left(S_{t, k} > S_t \pi_{t, k} \left(1+\frac{\epsilon}{\pi_{t, k}} \right) \, \bigg| \, S_t \right) 
  +\Pr \left(S_{t, k} < S_t \pi_{t, k} \left(1-\frac{\epsilon}{\pi_{t, k}} \right) \, \bigg| \, S_t\right) \\
  &\qquad= \Pr \left(S_{t, k} > \bbE[S_{t,k} \given S_t] \left(1+\frac{\epsilon}{\pi_{t, k}} \right) \right) 
   +\Pr \left(S_{t, k} < \bbE[S_{t,k} \given S_t] \left(1-\frac{\epsilon}{\pi_{t, k}} \right) \right)
\end{align}
As in Lemma~\ref{lem:consistency}, we have used the fact
that~$S_{t, k} \given S_t \sim \distBinomial(S_t, \pi_{t, k})$,
and hence has expectation~$S_t \pi_{t, k}$. 
The probability of this binomial random variable exceeding its mean 
by a multiplicative constant is a decreasing function of the 
number of spikes,~$S_t$. This implies that there exists a minimum 
number of trials~$S^*$ such that for~$S_t \geq S^*$, this probability 
of error is bounded above by~$\delta$, hence proving the first part 
of the theorem.

Now suppose~$S_t=S^*$. Since a binomial random variable,~$S_{t}$, can be seen
as a sum of independent coin flips, we can use a Chernoff bound to
upper bound the probability of deviating from the mean by a
multiplicative factor. We
have,
\begin{align}
  \Pr \left(S_{t,k} > \bbE[S_{t,k} \given S^*] \left(1+\frac{\epsilon}{\pi_{t,k}} \right) \right) 
  &\leq \exp \left \{- \frac{S^* \epsilon^2}{3 \pi_{t,k}} \right \},
\end{align}
and
\begin{align}
  \Pr \left(S_{t,k} < \bbE[S_{t,k} \given S^*] \left(1-\frac{\epsilon}{\pi_{t,k}} \right) \right) 
  &\leq \exp \left \{- \frac{S^* \epsilon^2}{2 \pi_{t,k}} \right \}.
\end{align}
Combining these, and leveraging the fact that~$\pi_{t,k} \leq 1$, we get,
\begin{align}
  \Pr(|\widehat{\pi}_{t,k} - \pi_{t,k} |  > \epsilon \given S^*) 
  & \leq 2 \exp \left \{- \frac{S_t \epsilon^2}{3} \right \}.
\end{align}
In fact, we can do even better (c.f. M\&U Exercise 4.13) and show,
\begin{align}
  \Pr(|\widehat{\pi}_{t,k} - \pi_{t,k} |  > \epsilon \given S^*) 
  &\leq 2 \exp \left \{-2S_t \epsilon^2 \right \}.
\end{align}

We bound the maximum deviation of any entry in~$\widehat{\bpi}$ with a union bound,
\begin{align}
  \Pr(\dtv(\widehat{\bpi}_t, \bpi_t) > \epsilon \given S^*) 
  & \leq 2K \exp \left\{-2S^* \epsilon^2 \right\}.
\end{align}
Setting this probability equal to~$\delta$ yields the desired bound on~$S^*$,
\begin{align}
  S^* &\leq \frac{1}{2\epsilon^2} \ln \frac{2K}{\delta}.
\end{align}

\end{proof}


This theorem provides an upper bound on the minimum number of spikes necessary to 
guarantee that the total variation distance between the true and estimated 
probability vectors is less than~$\epsilon$ with probability~$1-\delta$. 
Notably, the relevant quantity is the number of spikes~$S_t$, rather than 
the number of neurons. Thus, there is some flexibility in how the probability 
is estimated: a small population of neurons could be measured over many time bins,
or a large population could be measured over a single time bin. Moreover, the 
population gain,~$\gamma$, could be varied to adjust the number of spikes 
per time bin. 


In practice, the number of spikes cannot be set directly. It, is a
Poisson random variable whose mean, from Eq.~\ref{eq:S_t_rate},
is~$\gamma R$: the population gain times the number of neurons per
outcome. This leads to the following theorem, which specifies a upper bound 
on the gain and number of neurons required to guarantee that the 
total variation distance is less than~$\epsilon$ with probability~$1-\delta$.

\begin{theorem}
  \label{thm:rate_bounds}
  Given a fixed probability vector~$\bpi_t$, firing
  rates~$\lambda_{t,n} = \gamma \pi_{t,k_n}^{(j_n)}$, a fixed error
  level~$\epsilon < 1$, and a desired confidence~$\delta < 1$, the
  probability of error is bounded
  by~$\Pr(\dtv(\widehat{\bpi}_t, \bpi_t) > \epsilon) < \delta$
  if~$\gamma R \geq \lambda^*$, where~$\lambda^*$ is at most,
  \begin{align}
    \lambda^* &\leq \frac{1}{1-e^{-2\epsilon^2}} \ln \frac{2K}{\delta}.  
  \end{align}
\end{theorem}

\begin{proof}
  We have, 
  \begin{align}
    \Pr(\dtv(\widehat{\bpi}_t, \bpi_t) > \epsilon) 
    &= \sum_{m=0}^\infty \Pr(S_t=m) \Pr(\dtv(\widehat{\bpi}_t, \bpi_t) > \epsilon \given S_t=m) \\
    &\leq \sum_{m=0}^\infty  \Pr(S_t=m) \times 2K \exp \left \{-2m\epsilon^2 \right\} \\
    &= 2K \bbE_{S_t} \left[ \exp \left \{ -2S_t \epsilon^2 \right \} \right] \\
    &= 2K \exp \left \{\lambda^* (e^{-2\epsilon^2}-1) \right\},
  \end{align}
  where the last line follows from moment generating function of~${S_t \sim \distPoisson(\lambda^*)}$.
  Setting this equal to~$\delta$ and solving for~$\lambda^*$ yields the stated bound.
\end{proof}

This bound states that for a fixed gain factor, the number of neurons
required to guarantee that the total variation distance between the
true and estimated distributions is bounded by~$\epsilon$ could grow
extremely rapidly as~$\epsilon$ goes to zero.

So far we have considered the estimated probability distribution
obtained by ``reading out'' the entire population of neurons. What if
we only observe a fraction of the population, as a neuron in a
downstream population might? The next corollary shows that as long as
the observed neurons are randomly chosen in a way that is independent
of the locations they encode, the previously derived bounds on the required 
number of observed spikes, or alternatively, the expected number of 
spikes as a function of the gain and the number of observed neurons, 
still hold. 

% Inference section
\section{Bayesian Inference with Neural Dynamics}
Two forces cause the encoded distributions to change over time. As
we interact with the world and receive new inputs, the probabilities
of the visible variables change to reflect the new
observations. Moreover, even for a fixed set of visible variable
assignments, the probabilities of hidden variables will change as we
perform inference. Bayesian inference is the process of computing the
posterior distribution over latent variables given the values of visible
inputs. Many inference algorithms are iterative. Thus, as the
algorithms execute, the instantaneous probability distribution will
vary. Next, we describe how a simple inference algorithm can be
implemented with biologically plausible neural dynamics.

We assume that as an organism receives new inputs from the world, it
updates its posterior distribution over the values of latent
variables. Doing so requires a probabilistic model that relates hidden
and observed variables via a joint probability distribution.  Whereas
in previous chapters we have considered directed graphical models,
here we assume that the probabilistic models implemented in the brain
are best described in terms of a \emph{factor graph},
\begin{align}
  \label{eq:probabilistic_model}
  p(\bz \given \btheta) &=
  \frac{1}{Z(\btheta)}
  \prod_{j \in \mcG} \phi(z_j \given \btheta) \,
  \prod_{i,j \in \mcG} \phi(z_i, z_j \given \btheta) \,
\end{align}
Here, the graph,~$\mcG$, specifies pairs of variables with
probabilistic dependencies.  Each unary factor,~$\phi(\cdot \given
\btheta)$ is a function that maps a variable assignment to a
nonnegative real number, and each pairwise factor,~$\phi(\cdot, \cdot
\given \btheta)$, is a function that maps a pair of assignments,
say~${(z_i=k, z_j=k')}$ to a nonnegative real number. The
normalizing constant,~$Z(\btheta)$, ensures that the joint probability
distribution sums to one.

The probabilistic model in Eq.~\ref{eq:probabilistic_model} reflects
a specific assumption about the types of dependency structures
neural populations can represent.

\begin{assumption}
  Neural populations only perform inference in probabilistic models
  that factor into the product of unary and pairwise dependencies. 
\end{assumption}

A typical model need not factor into pairwise terms. It may instead
have factors that relate three or more latent variables. As we will
see, unary and pairwise factors map naturally onto neural biases
and synaptic weights. In our proposed neural implementation, higher
order factors would require the interaction of three or more neurons.
While this may be realized with dendritic computation or interneurons,
these more sophisticated implementations are beyond the scope of this
chapter. 

In general, the posterior distribution of a subset of hidden
variables~$\bz_H \subseteq \bz$ 
given the observed variables~$\bz_O = \bz \setminus \bz_H$ is,
\begin{align}
  p(\bz_H \given \bz_O, \btheta) &=
  \frac{p(\bz_H, \bz_O \given \btheta)}
       {\sum_{\bz_H} p(\bz_H, \bz_O \given \btheta)},
\end{align}
cannot be efficiently computed since it requires a sum over all
possible hidden variable assignments.
However, as we have seen in previous chapters, there are many methods of
approximating posterior distributions. Mean field variational
inference  maps particularly nicely onto the natural constraints
of neural dynamics. 
In mean field variational inference, the exact (but intractable)
posterior distribution is approximated with a factorized
distribution,
\begin{align}
  p(\bz_H \given \bz_O, \btheta) &\approx q(\bz_H) \equiv \prod_{z_j \in \bz_H} q(z_j).
\end{align}
The terms in this product are called \emph{variational factors}.  We
solve for the variational factors that minimize KL-divergence between
the true and approximate posterior,~$\KL(q(\bz_H) \,||\, p(\bz_H
\given \bz_O, \btheta))$. In minimizing the KL-divergence, we
simultaneously maximize a lower bound on the marginal log
likelihood,~$\log p(\bz_O \given \btheta)$.

The simplest method of minimizing this objective is via coordinate
descent, iteratively updating the probability of one hidden variable
given the probabilities of the rest. Since our variational
distribution is factorized, the variational factor for variable~$j$
must satisfy the mean field consistency equation:
\begin{align}
  \label{eq:mf_consistency}
  \log q(z_j) &\simeq \bbE_{q(\bz_{\neg j})}
  \left[ \log p(\bz_H, \bz_O \given \btheta) \right],
\end{align}
where~$\simeq$ denotes equality up to an additive constant and the
expectations are taken with respect to the variational
distribution over other hidden variables,
\begin{align}
  q(\bz_{\neg j}) &= \prod_{i \neq j} q(z_{i}).
\end{align}
The additive constant
ensures normalization of the probabilities, and will be discussed
subsequently.  

For discrete random variables, the variational factors are simply
vectors specifying the posterior probability of each variable,~$z_j$.
Under the direct representation described above, these instantaneous
values of these factors are encoded in the relative spike counts of
populations of neurons,
\begin{align}
  q_t(z_j=k) &= \widehat{\pi}_{t,k}^{(j)} 
\end{align}
To perform inference, the neuronal dynamics must be such that at each
time step, the relative spike counts satisfy~Eq.~\ref{eq:mf_consistency}. 
%For discrete random variables, this implies that the rate of a neuron~$n$
%that represents the variable-value pair~(${j_n=j,\,k_n=k}$) should obey,
Explicitly writing the additive constant, we have,
\begin{align}
  \nonumber
  \log \widehat{\pi}_{t,k}^{(j)} 
  &= -\log c_t^{(j)} + \log \phi(z_j=k \given \btheta) \\
  &\hspace{4em} + \bbE_{q_{t-1}(\bz_{\neg j})} \bigg[
  \sum_{i \in \neigh(j)} \log \phi(z_{i}, z_j=k \given \btheta) \bigg] \\
  % In terms of firing rates
  & \nonumber = 
     - \log c_t^{(j)} + \log \phi(z_j=k \given \btheta) \\
  & \label{eq:variational_rate}
  \hspace{4em} + \sum_{i \in \neigh(j)} \sum_{k'=1}^K
  \bigg[\log \phi(z_{i}=k', z_j=k \given \btheta) \,\cdot  \widehat{\pi}_{t-1,k'}^{(i)} \bigg] \\
  &= -\log c_t^{(j)} + \psi_{t,k}^{(j)}.
\end{align}
where we have combined the terms from the unary and pairwise factors
into the activation,~$\psi_{t,k}^{(j)}$.

Since~$\widehat{\bpi}_{t}^{(j)}$ is a probability distribution, we
know it must sum to one. Thus the additive constant must be set to
ensure this normalization.  Thus,
\begin{align}
  \widehat{\pi}_{t,k}^{(j)} &=
  \exp \left \{\psi_{t,k}^{(j)} -\log c_{t}^{(j)} \right \},
  & & &
  c_t^{(j)} &= \sum_{k'} \exp \left \{\psi_{t,k'}^{(j)} \right \}.
\end{align}

%Note that the pairwise factor functions are not necessarily symmetric.
%In fact, the two variables may have support for different ranges of
%values, so symmetry often does not make sense.  We have written this
%such that~$z_j$ is always the second argument, but its order will
%depend on its role in each factor.

Now that we have derived theoretically exact mean field updates, we
must show how they can be approximated with plausible neural dynamics.
We assume that inference occurs on a characteristic time scale of~$T_I$
time steps. This reflects the window of time over which neurons estimate
probability distributions.
From Lemma~\ref{lem:consistency}, we know that if the firing rates of
neurons the variable-value pair~${(z_j,k)}$ are proportional
to~$\widehat{\pi}_{t,k}^{(j)}$, then in expectation, the empirical
distribution represented by the spike counts will be equal to the
desired distribution. Thus, we aim to set,
\begin{align}
  \lambda_{t,n} &= \gamma \, \widehat{\pi}_{t,k_n}^{(j_n)}
  = \gamma \exp \left \{\psi_{t,k_n}^{(j_n)} -\log c_{t}^{(j_n)} \right \},
\end{align}
for fixed \emph{gain},~$\gamma$. While this rate function is
nearly a linear-nonlinear cascade, as we studied in previous
chapters, there is one major impediment to realizing this
calculation in biological neurons. Specifically, to compute
the activation, a neuron must have access to the \emph{normalized}
probabilities of other hidden and visible variables. In practice,
a neuron only observes the spike counts, which are
unnormalized probabilities. This motivates our next assumption,

\begin{assumption}
  All neurons in the population share the same gain,~$\gamma$.
  Thus,
  \begin{align}
    \bbE \left[ \sum_{n=1}^N \bbI[j_n=j] \, s_{t,n} \right]
    &= \bbE \left[ \sum_{n=1}^N \sum_{k=1}^K \bbI[j_n=j] \, \bbI[k_n=k] \,s_{t,n} \right] \\
    &= \gamma \sum_{n=1}^N \sum_{k=1}^K \bbI[j_n=j] \, \bbI[k_n=k] \, \widehat{\pi}_{t,k}^{(j)} \\
    &= \gamma R.
  \end{align}

  Moreover, the instantaneous probability is well-approximated by,
  \begin{align}
      \widehat{\pi}_{t,k}^{(j)} &=
  \frac{\sum_{\Delta=1}^{T_I} \sum_{n=1}^N \bbI[j_n=j] \, \bbI[k_n=k] \, s_{t-\Delta ,n}}
       {\sum_{\Delta=1}^{T_I} \sum_{n=1}^N \bbI[j_n=j] \, s_{t-\Delta ,n}} \\
       &\approx (\gamma T_I R)^{-1} \sum_{\Delta=1}^{T_I} \sum_{n=1}^N \bbI[j_n=j] \, \bbI[k_n=k] \, s_{t-\Delta,n}.
  \end{align}
  In other words, the total spike count is concentrated around its mean.
\end{assumption}

Under the assumption of shared gain, the
desired dynamics in~Eq.~\ref{eq:variational_rate}
simplify to,
\begin{align}
  \label{eq:mf_log_probs}
  \lambda_{t,n} &= \gamma
  \exp \left \{b_n + (\gamma T_I R)^{-1} \sum_{\Delta=1}^{T_I} \sum_{m=1}^N w_{n \from m} \cdot s_{t-\Delta,m}
  - \log c_t^{(j_n)} \right \}
\end{align}
where
\begin{align}
  b_n &= \log \phi(z_{j_n} = k_n \given \btheta),
\end{align}
and
\begin{align}
  w_{n \from m} &=
  \begin{cases}
    \log \phi(z_{j_n}=k_n, z_{j_m}=k_m \given \btheta) & \text{if }j_m
    \in \neigh(j_n) \\
    0 & \text{o.w.}
  \end{cases}
\end{align}

\TODO{Show a figure of inference dynamics for a simple model}

The last step is to compute the normalizing input,~$c_t^{(j)}$.
This requires summing the instantaneous rates of all neurons representing the
random variable,~$z_j$. While this is clearly implausible, we may derive
a gain controller from an alternative perspective. Normalizing the
probability distribution ensures that the expected spike count at
any time step for
neurons representing~$z_j$ is equal to~$R\gamma$. If the distribution
is not properly normalized, the expected spike count will deviate.
Thus, a reasonable gain controller can estimate the population
can estimate the population rate,
\begin{align}
\widehat{\lambda}_{t}^{(j)} &= \sum_{\Delta = 1}^{T_G} \sum_{n=1}^N \bbI[j_n=j] s_{t-\Delta, n},
\end{align}
and set the control input to,
\begin{align}
  c_{t}^{(j)} &= \frac{\widehat{\lambda}_{t}^{(j)}}{\gamma T_G R}.
\end{align}
The time scale of the gain controller is typically set to be
less than the time scale of inference, that is~$T_G < T_I$.


This theory provides a normative interpretation of synaptic weights.
Here, synaptic weights reflect the conditional log probabilities
for the variable-value pairs represented by the pre- and post-synaptic
neurons. 

\begin{comment}
\section{Unsupervised Learning via Synaptic Plasticity}
\label{sec:learning}
The parameters of the model,~$\btheta$, specify the conditional
probabilities for pairs of hidden and visible variables. Rather than
treating the parameters as given, we now treat them as part of the
model.
\begin{align}
  p(\bx, \bz, \btheta) &= p(\btheta) \, p(\bx, \bz \given \btheta)  \\
  &= \frac{p(\btheta)}{Z(\btheta)} \prod_{h \in \mcG} \phi(z_j \given \btheta) \prod_{h,h' \in \mcG} \phi(z_j, z_{h'} \given \btheta.
\end{align}
The challenge with learning is that the parameters appear in the
normalizing constant,~$Z(\btheta)$, which is typically an intractable
summation over variable assignments. For the purposes of this chapter,
we will only consider learning in a subset of models that can
formulated as directed graphical models.

\begin{assumption}
  The following learning algorithm assumes that the probabilistic
  model not only factors into the product of unary and pairwise
  potentials, but that this factorization corresponds to a directed
  graphical model in which the variables have at most one ``parent''
  variable. That is, the variables are ordered such that the
  joint probability is equal to,
  \begin{align}
    p(\bz, \btheta) &= p(\btheta) \prod_{j=1}^J p(z_j \given \pa(z_j), \btheta),
  \end{align}
  where~$\pa(z_j) \in \{z_1, \ldots, z_{j-1}\} \cup \varnothing$.
  Each conditional distribution in this product is properly normalized,
  which implies that the joint distribution is normalized as well.
\end{assumption}

While this is clearly a strict assumption, we will see that it allows
for some realistic models. The advantage is that, here, the
distribution is normalized such that the parameters appear only
in their prior and in the conditional distributions, which depend
on at most two variables. This will map nicely onto synaptic plasticity
rules. 

Since we are assuming the variables
are discrete, the parameters~$\btheta$ specify either the marginal
probability of~$z_j$ (if~$\pa(z_j)=\varnothing$) or the rows of a
conditional probability table (if~$\pa(z_j) \in \{z_1, \ldots, z_{j-1}\}$).
We
make this explicit with the following notation,
\begin{align}
  p(z_j \given \pa(z_j)=\varnothing, \btheta) &= \distCategorical(\btheta^{(j)}), \\
  p(z_j \given \pa(z_{j})=z_{j'}=k, \btheta) &=  \distCategorical(\btheta^{(j,k)}).
\end{align}
In words, if the variable~$z_j$ has no parent, it is marginally distributed
according to a categorical distribution with parameter~$\btheta^{(j)}$.
If variable~$z_j$ has parent~$z_{j'}$, then when~$z_{j'}=k$, the
variable~$z_j$ follows a categorical distribution with
parameter~$\btheta^{(j,k)}$. 

To incorporate these parameters into the model, we introduce Dirichlet priors
over the probability vectors,
\begin{align}
  \btheta^{(j)} &\sim \distDirichlet(\alpha \bone), & & &
  \btheta^{(j,k)} &\sim \distDirichlet(\alpha \bone).
\end{align}

Learning in a Bayesian framework corresponds to performing posterior
inference over the parameters. Thus, we introduce a variational factor
for~$\btheta$ as well,
\begin{align}
  q(\btheta) &=
  \prod_{j: \pa(z_j)=\varnothing} q(\btheta^{(j)})
  \prod_{j: \pa(z_j)\neq \varnothing} \prod_{k} q(\btheta^{(j,k)})
\end{align}
Now consider the mean field consistency equation governing the parameter
of a root variable (with no parent),
\begin{align}
  \log q_t(\btheta^{(j)}) &\simeq
  \bbE_{q_{t-1}(\bz)}\left[ \log p(\bz, \btheta) \right] \\
  &\simeq \bbE_{q_{t-1}(\bz)}
  \left[ \sum_{k=1}^K \bbI[z_j=k] \log \theta_k^{(j)} \right]
  + \sum_{k=1}^K (\alpha-1) \log \theta_k^{(j)}  \\
  &\simeq  \sum_{k=1}^K (\widehat{\pi}_{t-1,k}^{(j)} + \alpha-1) \log \theta_k^{(j)}
\end{align}
This is the form of a gamma distribution, which implies,
\begin{align}
  q_t(\btheta^{(j)})
  &= \distDirichlet(\btheta^{(j)} \given  \balpha_{t}^{(j)}) \\
  \balpha_{t}^{(j)} &= \alpha + \widehat{\bpi}_{t-1}^{(j)}
\end{align}
Similarly, the variational factors for~$\btheta^{(j,k)}$
follows a Dirichlet distribution as well,
\begin{align}
  q_t(\btheta^{(j,k)}) &=
  \distDirichlet(\btheta^{(j,k)} \given \balpha_{t}^{(j,k)}), \\
  \balpha_{t}^{(j,k)} &= \alpha + \widehat{\bpi}_{t-1}^{(j)} \cdot \widehat{\pi}_{t-1,k}^{(\pa(z_j))}.
  %q_t(\theta_{k,k'}^{(v,h)}) &=
  %\distGamma(\theta_{k,k'}^{(v,h)} \given
  %\alpha + \widehat{\pi}_{t-1,k}^{(v)} \cdot \widehat{\pi}_{t-1,k'}^{(j)},
  %\, \beta).
\end{align}

Returning to the task of inferring the posterior distribution over
hidden variables, we see that the updates for~$q(z_j)$ must now
be derived with respect to the expectation of~$q(\btheta)$ as well,
\begin{align}
  \log q(z_j) &\simeq \bbE_{q(\bz_{\neg j})} \bbE_{q(\btheta)} \left[ p(\bz, \btheta) \right].
\end{align}
Instead of working directly with~$\log p(z_j \given \pa(z_j), \btheta)$,
the updates must work with its expectation under the Dirichlet
variational factor. If~$\pa(z_j)=\varnothing$, we have
\begin{align}
  \bbE_{q_t(\btheta)} \left[\log p(z_j=k \given \pa(z_j)=\varnothing, \btheta)\right]
  &= \bbE_{q_t(\btheta)} \left[\log \theta_{k}^{(j)} \right] \\
  \label{eq:expected_log_theta}
  &= \psi\big(\alpha_{t,k}^{(j)}\big)
  - \psi\big(\sum_{i=1}^K \alpha_{t,i}^{(j)} \big) \\
  &= \psi\big(\alpha_{t,k}^{(j)}\big)
  - \psi\big(\sum_{i=1}^K \alpha + \widehat{\pi}_{t-1,i}^{(j)} \big) \\
  &= \psi\big(\alpha_{t,k}^{(j)}\big)
  - \psi\big(K\alpha\big).
\end{align}
where~$\psi(\cdot)$ is the digamma function. Otherwise,
\begin{align}
  \bbE_{q_t(\btheta)} \left[\log p(z_j=k \given \pa(z_j)=k', \btheta)\right]
  &= \bbE_{q_t(\btheta)} \left[\log \theta_{k}^{(j,k')} \right] \\
  \label{eq:expected_log_theta}
  &= \psi\big(\alpha_{t,k}^{(j,k')}\big)
  - \psi\big(\sum_{i=1}^K \alpha_{t,i}^{(j,k')} \big) \\
  &= \psi\big(\alpha_{t,k}^{(j,k')}\big)
  - \psi\big(\sum_{i=1}^K \alpha + \widehat{\pi}_{t-1,i}^{(j)} \widehat{\pi}_{t-1,k'}^{(\pa(z_j)} \big) \\
  &= \psi\big(\alpha_{t,k}^{(j,k')}\big)
  - \psi\big(K\alpha + \widehat{\pi}_{t-1,k'}^{(\pa(z_j)} \big).
\end{align}

How could this be implemented biologically?
First, we assume that learning occurs on a 
timescale of~$T_L$ time steps, which is relatively slow compared to
the time scales of inference and behavior. That is,~$T_I < T_L$.
This allows the learning algorithm to generalize from many
input rather than overfitting to a single example.

For root variables,~$\theta_k^{(j)}$ sets an activation bias that sets
the baseline firing rate. We assume that these neurons have a dynamic
state variable,~$\alpha_{t,n}$, that roughly computes a moving average
of its firing rate,
\begin{align}
  \alpha_{t,n} &= \alpha + (\gamma T_L)^{-1} \cdot \sum_{\Delta = 1}^{T_L} s_{t-\Delta ,n} \\
  &\approx \alpha + \widehat{\pi}_{t,k_n}^{(j_n)}
\end{align}
This state variable governs the instantaneous activation bias,
\begin{align}
  b_{t,n} &= \psi \big(\alpha_{t,n} \big) - \psi \big( K\alpha \big).
\end{align}
If a neuron tends to have a high firing rate, reflecting a high 
marginal posterior probability, this will eventually be learned and
incorporated into the activation bias.



We want the synaptic weights to equal the expected log parameter
value, as in Eq.~\ref{eq:expected_log_theta}.  In theory, the weights
should be identical for all synapses between neurons
representing~$(z_j=k)$ and neurons representing~$(z_{h'}=k')$.  It is
unreasonable to assume this in practice, since these synapses exist on
different neurons and are updated independently. However, we can
specify a simple learning rule that would give rise to the same
weights in expectation.

Assume that each synapse has a two latent states,~$\alpha_{t, n \from m}$,
and~$\beta_{t,n \from m}$. These will enable us to compute
the expectation with respect to the variational parameter.
We propose the following learning rule for the first state,
\begin{align}
  \alpha_{t, n \from m} &=
  \alpha +
  (\gamma^2 T_L)^{-1}  \sum_{\Delta = 1}^{T_L} s_{t-\Delta ,n} \cdot s_{t-\Delta,m} \\
  &\approx \alpha + \widehat{\pi}_{t-1,k_n}^{(j_n)} \cdot \widehat{\pi}_{t-1,k_m}^{(h_m)}. 
\end{align}
The tricky, asymmetric part comes in whether the is on the
child neuron,~$z_{h_m} = \pa(z_{j_n})$, or the parent
neuron,~$z_{j_n}= \pa(z_{h_m})$. This determines how the second
state must be computed. If~$z_{j_n}$ is the child, then
\begin{align}
  \beta_{t,n \from m} &= K\alpha + (\gamma T_L)^{-1} \sum_{\Delta = 1}^{T_L} s_{t -\Delta, m} \\
  &\approx K\alpha + \widehat{\pi}_{t-1,k_m}^{(z_{h_m})}.
\end{align}
Otherwise, if~$z_{j_n} = \pa(z_{h_m})$,
\begin{align}
  \beta_{t,n \from m} &= K\alpha + (\gamma T_L)^{-1} \sum_{\Delta = 1}^{T_L} s_{t -\Delta, n} \\
  &\approx K\alpha + \widehat{\pi}_{t-1,k_n}^{(z_{j_n})}.
\end{align}
The difference is in whether the second state counts pre- or post-synaptic
spikes. 

The synaptic weight is a deterministic function of these two state variables,
\begin{align}
  w_{t, n \from m} &= \psi(\alpha_{t, n \from m}) - \psi(\beta_{t, n \from m}).
\end{align}
This state-based learning rule is Hebbian in that correlated spiking
activity leads to increases in~$\alpha_{t,n\from m}$, which in turn
lead to larger weights (since the digamma function is increasing on
the non-negative reals).
This is counteracted by the accrual of~$\beta_{t, n \from m}$, which
counts pre- or post-synaptic spikes, depending on whether the
post-synaptic neuron represents the child or parent variable, respectively.
If this value is large relative to~$\alpha_{t, n \from m}$, the
spike correlation is low relative to the background rate, which
implies a low probability and a strongly negative weight. 

Moreover, this learning rule is nonlinear. While the state variables
are linear functions of pre- and post-synaptic spike counts, their
effect on the weight is highly nonlinear due to the digamma functions.
Finally, we could instead write this learning rule as a nonlinear
dynamical system on the weights alone since the digamma function is
also invertible on this range. We leave this for future work.
\end{comment}

\section{Examples}


\subsection{Simultaneous Localization and Mapping}

Model:
\begin{itemize}
\item Latent locations,~$z_i \in \{1, \ldots, K\}$ for time indices~$-D, \ldots, 0$.
\item Sensory cues,~$x_{i,j} \in \{0,1\}$ for~$j \in \{1, \ldots, J\}$.
\item Transition probability~$\phi(z_{i-1}, z_i)$. How could these weights be shared?
\item Observation probability~$\phi(z_{i}, x_{i,j})$. And how could these be shared?
\end{itemize}

\subsection{Olfactory Scene Parsing}


Model:
\begin{itemize}
\item Presence or absence of odorant~$n$ in scene~$d$:~$w_{d,n} \in \{0,1\}$
\item Presnce or absence of object~$k$ in scene~$d$:~$\eta_{d,k} \in \{0,1\}$
\item Object that caused odorant~$n$ in scene~$d$: $z_{d,n} \in \{1, \ldots, K\}$ 
\item Baseline probability of object~$k$:~$\phi(\eta_{d,k})$
\item Probability of object~$k$ in scene~$d$ causing odorant~$n$:~$\phi(z_{d,n}=k, \eta_{d,k})$
\item Probability of odorant~$n$ given object~$k$:~$\phi(w_{d,n}, z_{d,n})$
\end{itemize}
