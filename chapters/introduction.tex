\chapter{Introduction}
\label{introduction}

% The confluence of advanced recording techniques for measuring neural 
% activity and scalable machine learning algorithms makes this a 
% particularly exciting time in computational neuroscience. In some organisms,
% we already have access to whole-brain recordings. In order to understand the 
% computations neural circuits implement, we need to approach this problem 
% from the both the top down and the bottom up. That is, we need 
% theories of neural computation that rest on principled foundations, and 
% we need statistical methods capable of instantiating these theories 
% in probabilistic models and testing them against large-scale brain recordings.
% This thesis is about building a suite of statistical models and computational 
% algorithms that expands the frontier of models that we can efficiently 
% instantiate and fit. In doing so, we continue to close the gap between our theory 
% of neural computation and our methods of analysis.

Neuroscience is experiencing a technological revolution. Whereas traditional 
methods were limited to either direct measurements of a handful of neurons 
or gross measurements of average activity in entire brain volumes, modern
imaging techniques and microelectrode arrays enable precise measurements
of thousands of neurons at a time. For some organisms, we can now
monitor the activity of every single neuron in the brain.  This unprecedented
capability is leading to a paradigm shift in the study of the brain. 
Brain states, 
 
