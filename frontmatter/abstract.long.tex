Neuroscience is entering an exciting new age.  Modern recording
technologies enable simultaneous measurements of
thousands of neurons in organisms performing complex behaviors.  Such
recordings offer an unprecedented opportunity to glean insight into
the mechanistic underpinnings of intelligence, but they also present an
extraordinary statistical and computational challenge: how do we make
sense of these large scale recordings?
This thesis develops a suite of tools that instantiate
hypotheses about neural computation in the form of
probabilistic models and a corresponding set of Bayesian inference
algorithms that efficiently fit these models to neural spike trains.
From the posterior distribution of model parameters and variables,
we seek to advance our understanding of how the brain works. 


At the core of our probabilistic models is a collection of
structural motifs, the recurring design patterns from which we
construct interpretable models. These include random network models,
which connect latent types and features of neurons to the dynamics of
spike trains, and state space models, which capture the dynamics of
neural data in terms of a latent state that evolves over time.  The
challenge lies in reconciling these models with the
discrete nature of neural spike trains.  To surmount this challenge, 
we build on the Hawkes process --- a
multivariate generalization of the Poisson process --- and its
discrete time analogue, the linear autoregressive Poisson model.  By
leveraging the linear nature of these models and the Poisson
superposition principle, we derive elegant auxiliary variable
formulations and efficient inference algorithms. We then generalize
these to nonlinear and nonstationary models of neural spike trains and
take advantage of the \polyagamma augmentation to develop novel
Markov chain Monte Carlo (MCMC) inference algorithms.

In the latter chapters, we shift our focus from autoregressive models
to latent state space models of neural activity. We perform an
empirical study of Bayesian nonparametric methods for hidden Markov
models of neural spike trains. Then, we leverage the
\polyagamma augmentation to develop an efficient MCMC algorithm for
switching linear dynamical systems with discrete observations.  In
pursuit of a principled means of Bayesian model comparison, we develop
an annealed importance sampling method for estimating the marginal
likelihood of complex spike train models. To make this method work in
practice, we devise a novel sampling algorithm for the \polyagamma
distribution. With this development, we can compare many of the models
presented in this thesis in a rigorous, Bayesian manner.

Finally, we consider the ``Bayesian brain'' hypothesis --- the
hypothesis that neural circuits are themselves performing Bayesian
inference.  We show how one particular implementation of this
hypothesis implies autoregressive dynamics of the form studied
in earlier chapters, thereby providing a theoretical interpretation of
our probabilistic models.  This closes the
loop, connecting top-down theory with bottom-up inferences, and
suggests a path toward translating large scale recording
capabilities into new insights about neural computation.


